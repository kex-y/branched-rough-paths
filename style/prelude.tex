\mathtoolsset{showonlyrefs=true}

\newif\ifdark
\IfFileExists{dark}{\darktrue}{\darkfalse}

\ifdark

\definecolor{darkred}{rgb}{0.9,0.2,0.2}
\definecolor{darkblue}{rgb}{0.7,0.3,1}
\definecolor{darkgreen}{rgb}{0.1,0.9,0.1}
\definecolor{pagebackground}{rgb}{.15,.21,.18}
\definecolor{pageforeground}{rgb}{.84,.84,.85}
\pagecolor{pagebackground}
\AtBeginDocument{\globalcolor{pageforeground}}

\else

\definecolor{darkred}{rgb}{0.7,0.1,0.1}
\definecolor{darkblue}{rgb}{0.4,0.1,0.8}
\definecolor{darkgreen}{rgb}{0.1,0.7,0.1}
\definecolor{pagebackground}{rgb}{1,1,1}
\definecolor{pageforeground}{rgb}{0,0,0}

\fi

% Comments
\definecolor{newred}{RGB}{208,16,76}
\definecolor{newgreen}{RGB}{34,125,81}
\def\zb#1{\comment[newred]{ZB: #1}}
\def\xr#1{\comment[newgreen]{XR:#1}}
\long\def\zbtext#1{{\color{newred}ZB:\ #1}}
\long\def\xrtext#1{{\color{newgreen}XR:\ #1}}

%\DeclarePairedDelimiter{\nint}\lfloor\rceil
%\DeclarePairedDelimiter{\abs}\lvert\rvert

\makeatletter
\newcommand{\globalcolor}[1]{%
	\color{#1}\global\let\default@color\current@color
}
\makeatother

\DeclareSymbolFont{timesoperators}{T1}{ptm}{m}{n}
\SetSymbolFont{timesoperators}{bold}{T1}{ptm}{b}{n}

\makeatletter
\renewcommand{\operator@font}{\mathgroup\symtimesoperators}
\makeatother

\DeclareMathAlphabet{\mathbbm}{U}{bbm}{m}{n}
\overfullrule=3mm
\marginparwidth=3.3cm

\DeclareFontFamily{U}{BOONDOX-calo}{\skewchar\font=45 }
\DeclareFontShape{U}{BOONDOX-calo}{m}{n}{
	<-> s*[1.05] BOONDOX-r-calo}{}
\DeclareFontShape{U}{BOONDOX-calo}{b}{n}{
	<-> s*[1.05] BOONDOX-b-calo}{}
\DeclareMathAlphabet{\mcb}{U}{BOONDOX-calo}{m}{n}
\SetMathAlphabet{\mcb}{bold}{U}{BOONDOX-calo}{b}{n}

\makeatletter 

% Stolen from the internet to make a fat \cdot which isn't as fat as a \bullet
\newcommand*{\fat}{}% Check if undefined
\DeclareRobustCommand*{\fat}{%
	\mathbin{\mathpalette\bigcdot@{}}}
\newcommand*{\bigcdot@scalefactor}{.5}
\newcommand*{\bigcdot@widthfactor}{1.15}
\newcommand*{\bigcdot@}[2]{%
	% #1: math style
	% #2: unused
	\sbox0{$#1\vcenter{}$}% math axis
	\sbox2{$#1\cdot\m@th$}%
	\hbox to \bigcdot@widthfactor\wd2{%
		\hfil
		\raise\ht0\hbox{%
			\scalebox{\bigcdot@scalefactor}{%
				\lower\ht0\hbox{$#1\bullet\m@th$}%
			}%
		}%
		\hfil
	}%
}

\DeclareRobustCommand{\TitleEquation}[2]{\texorpdfstring{\StrLeft{\f@series}{1}[\@firstchar]$\if b\@firstchar\boldsymbol{#1}\else#1\fi$}{#2}}

\makeatother

%\def\newoptest#1{\expandafter\DeclareMathOperator\csname#1\endcsname{#1}}
%\newoptest{Ciao}

\theoremnumbering{Alph}
\renewtheorem{theorem*}{Theorem}
\newtheorem{assumption}[lemma]{Assumption}
\newtheorem{example}[lemma]{Example}
\numberwithin{equation}{section}

\def\restr{\mathord{\upharpoonright}}
\def\slash{\leavevmode\unskip\kern0.18em/\penalty\exhyphenpenalty\kern0.18em}
\def\dash{\leavevmode\unskip\kern0.18em--\penalty\exhyphenpenalty\kern0.18em}

%%%%%%%%%%%%%%%Notation%%%%%%%%%%%%%%%
\def\emptyset{{\centernot\Circle}}
\let\epsilon\varepsilon
\def\Ito{{\textnormal{\tiny Itô}}}
\def\${|\!|\!|}
\def\cst#1{\big(H-{\textstyle {#1\over 2}}\big)}
\def\RKHS{\normalfont\textsc{rkhs}}
\setlist{noitemsep,topsep=4pt}
\def\para_#1{/\!\!/_{\!#1}}
\def\PPi{\boldsymbol{\Pi}}
\newcommand{\Anorm}[2]{\| #1 \|^{(2)}_{#2}}
\newcommand{\Avert}[2]{\|| #1 |\|^{(2)}_{#2}}
\newcommand{\de}{\overset{\mathrm{def}}{=}}
