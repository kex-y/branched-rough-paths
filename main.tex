\documentclass[11pt]{style/preprint}
\usepackage{style/comment}
\usepackage{style/mhenvs}
\usepackage{style/mhsymb}
\usepackage{style/trees}

\usepackage[full]{textcomp}
\usepackage[osf]{newtxtext}
\usepackage{amssymb}
\usepackage{amsmath}
\usepackage{mathtools}
\usepackage{mhequ}
\usepackage{booktabs}
\usepackage{tikz}
\usepackage{mathrsfs}
\usepackage{longtable}
\usepackage{microtype}
\usepackage{wasysym}
\usepackage{centernot}
\usepackage{enumitem}
\usepackage[dvipsnames]{xcolor}
\usepackage{hyperref}
\hypersetup{
  pdftitle={},
  pdfauthor={Kexing Ying},
  colorlinks=true,
  linkcolor=Maroon,
  filecolor=Maroon,
  citecolor=Maroon,
  urlcolor=Maroon,
  pdfcreator={LaTeX via pandoc}}
\usepackage{crossreftools}
\usepackage{physics}
\usepackage[notref,notcite]{showkeys}

\mathtoolsset{showonlyrefs=true}

\newif\ifdark
\IfFileExists{dark}{\darktrue}{\darkfalse}

\ifdark

\definecolor{darkred}{rgb}{0.9,0.2,0.2}
\definecolor{darkblue}{rgb}{0.7,0.3,1}
\definecolor{darkgreen}{rgb}{0.1,0.9,0.1}
\definecolor{pagebackground}{rgb}{.15,.21,.18}
\definecolor{pageforeground}{rgb}{.84,.84,.85}
\pagecolor{pagebackground}
\AtBeginDocument{\globalcolor{pageforeground}}

\else

\definecolor{darkred}{rgb}{0.7,0.1,0.1}
\definecolor{darkblue}{rgb}{0.4,0.1,0.8}
\definecolor{darkgreen}{rgb}{0.1,0.7,0.1}
\definecolor{pagebackground}{rgb}{1,1,1}
\definecolor{pageforeground}{rgb}{0,0,0}

\fi

% Comments
\definecolor{newred}{RGB}{208,16,76}
\definecolor{newgreen}{RGB}{34,125,81}
\def\zb#1{\comment[newred]{ZB: #1}}
\def\xr#1{\comment[newgreen]{XR:#1}}
\long\def\zbtext#1{{\color{newred}ZB:\ #1}}
\long\def\xrtext#1{{\color{newgreen}XR:\ #1}}

%\DeclarePairedDelimiter{\nint}\lfloor\rceil
%\DeclarePairedDelimiter{\abs}\lvert\rvert

\makeatletter
\newcommand{\globalcolor}[1]{%
	\color{#1}\global\let\default@color\current@color
}
\makeatother

\DeclareSymbolFont{timesoperators}{T1}{ptm}{m}{n}
\SetSymbolFont{timesoperators}{bold}{T1}{ptm}{b}{n}

\makeatletter
\renewcommand{\operator@font}{\mathgroup\symtimesoperators}
\makeatother

\DeclareMathAlphabet{\mathbbm}{U}{bbm}{m}{n}
\overfullrule=3mm
\marginparwidth=3.3cm

\DeclareFontFamily{U}{BOONDOX-calo}{\skewchar\font=45 }
\DeclareFontShape{U}{BOONDOX-calo}{m}{n}{
	<-> s*[1.05] BOONDOX-r-calo}{}
\DeclareFontShape{U}{BOONDOX-calo}{b}{n}{
	<-> s*[1.05] BOONDOX-b-calo}{}
\DeclareMathAlphabet{\mcb}{U}{BOONDOX-calo}{m}{n}
\SetMathAlphabet{\mcb}{bold}{U}{BOONDOX-calo}{b}{n}

\makeatletter 

% Stolen from the internet to make a fat \cdot which isn't as fat as a \bullet
\newcommand*{\fat}{}% Check if undefined
\DeclareRobustCommand*{\fat}{%
	\mathbin{\mathpalette\bigcdot@{}}}
\newcommand*{\bigcdot@scalefactor}{.5}
\newcommand*{\bigcdot@widthfactor}{1.15}
\newcommand*{\bigcdot@}[2]{%
	% #1: math style
	% #2: unused
	\sbox0{$#1\vcenter{}$}% math axis
	\sbox2{$#1\cdot\m@th$}%
	\hbox to \bigcdot@widthfactor\wd2{%
		\hfil
		\raise\ht0\hbox{%
			\scalebox{\bigcdot@scalefactor}{%
				\lower\ht0\hbox{$#1\bullet\m@th$}%
			}%
		}%
		\hfil
	}%
}

\DeclareRobustCommand{\TitleEquation}[2]{\texorpdfstring{\StrLeft{\f@series}{1}[\@firstchar]$\if b\@firstchar\boldsymbol{#1}\else#1\fi$}{#2}}

\makeatother

%\def\newoptest#1{\expandafter\DeclareMathOperator\csname#1\endcsname{#1}}
%\newoptest{Ciao}

\theoremnumbering{Alph}
\renewtheorem{theorem*}{Theorem}
\newtheorem{assumption}[lemma]{Assumption}
\newtheorem{example}[lemma]{Example}
\numberwithin{equation}{section}

\def\restr{\mathord{\upharpoonright}}
\def\slash{\leavevmode\unskip\kern0.18em/\penalty\exhyphenpenalty\kern0.18em}
\def\dash{\leavevmode\unskip\kern0.18em--\penalty\exhyphenpenalty\kern0.18em}

%%%%%%%%%%%%%%%Notation%%%%%%%%%%%%%%%
\def\emptyset{{\centernot\Circle}}
\let\epsilon\varepsilon
\def\Ito{{\textnormal{\tiny Itô}}}
\def\${|\!|\!|}
\def\cst#1{\big(H-{\textstyle {#1\over 2}}\big)}
\def\RKHS{\normalfont\textsc{rkhs}}
\setlist{noitemsep,topsep=4pt}
\def\para_#1{/\!\!/_{\!#1}}
\def\PPi{\boldsymbol{\Pi}}
\newcommand{\Anorm}[2]{\| #1 \|^{(2)}_{#2}}
\newcommand{\Avert}[2]{\|| #1 |\|^{(2)}_{#2}}
\newcommand{\de}{\overset{\mathrm{def}}{=}}


\begin{document}

\title{Branched Rough Path Notes\\
  \large Based on \href{arxiv.org/abs/1210.6294}{arxiv.org/abs/1210.6294}}
\author{}
\institute{}
\maketitle

% \begin{equ}
% \fT = \{\<1>,\<2>,\<2>_x,\<20>,\<3>,\<30>, \<22>, \<31>,\<32>\}\;,
% \end{equ}

\section{Trees}

% Whilst for level 2 rough paths (i.e. \(\alpha \in \left(\frac{1}{3}, \frac{1}{2}\right]\)), we can 
% get away with writing out the the algebraic and analytic conditions explicitly, this quickly becomes 
% unbearable for higher level rough paths. Instead, by using the language of trees, it turns out 
% that these conditions can be written neatly into a single equation.

For higher order rough paths, it is useful to introduce tree notations to keep track of the 
algebraic and analytic conditions for the extra terms arising from the Taylor expansion.

\begin{definition}
  Denote \(\CT\) for the set of all rooted and labeled trees and \(\CF\) for the set of the associated 
  forests, i.e. \(\CF\) is the free abelian monoid generated buy \(\CT\) and \(f \in \CF\) is a finite 
  collection of rooted labeled trees (where we allow multiple copies of the same tree).
\end{definition}

Some examples of rooted and labeled trees are 
\[\<X>, \<1>, \<2>, \<10>, \<3>, \<20>, \<100>, \cdots\]
We write \(f = \prod_{i \in I} h_i \in \CF\) as a commutative product where \(h_i \in \CT\) and \(I\) 
is a finite index set. We allow \(\CT\) to contain the empty tree and we denote it by \(\mathbf{1}\) 
since it is the multiplicative identity in \(\CF\). We introduce the rooting operation \([-]_a : \CF \to \CT\) where 
for \(f \in \CF\), we define \([f]_a\) for the tree obtained by joining all the trees in \(f\) to the
root \(a\), e.g. 
\[[\mathbf{1}]_a = \<X> \text{ and } [\<X> \<1cb>]_d = \<X10>.\]
For \(h \in \CT\), denote \(f_h\) for the number of copies of \(h\) in \(f\). Thus, we can write 
\(f = \prod_{h \in \CT} h^{f_h}\)
where \(f_h\) is zero for all but finitely many terms. Namely, \(h \in f\) if and only if 
\(f_h \neq 0\). For \(h \in \CT\), the order of \(h\): \(|h|\) is simply the number of vertices of 
\(h\) and for \(f \in \CF\), \(|f| = \sum_{h \in \CT} f_h |h|\). We also write \(\#f = \sum_{h \in \CT} f_h\).

We denote \(\langle \CF \rangle\) for the free vector space generated by \(\CF\) so \(\langle \CF\rangle\) 
is a unital commutative algebra with identity \(\mathbf{1}\). We define the inner product of \(\langle \CF \rangle\)
by setting 
\begin{equation}\label{eq:inner-product}
  \langle f, f'\rangle = \mathbf{1}_{f = f'}
\end{equation}
for all \(f, f' \in \CF\) and extending linearly. 

We define the coproduct \(\Delta : \langle \CF \rangle \to \langle \CF \rangle \otimes \langle \CF \rangle\) 
by first defining \(\Delta\) on \(\CF\) inductively and then extending linearly. In particular, we set
\begin{itemize}
  \item \(\Delta \mathbf{1} = \mathbf{1} \otimes \mathbf{1}\).
  \item For \(f \in \CF\), \(\Delta f = \prod_{h \in \CT} (\Delta h)^{f_h}\).
  \item For \(f \in \CF\), \(\Delta [f]_a = [f]_a \otimes \mathbf{1} + (\text{id} \otimes [-]_a) \Delta f\).
\end{itemize} 
Here are some examples:
\begin{itemize}
  \item \(\Delta \<X> = \<X> \otimes \mathbf{1} + \mathbf{1} \otimes \<X>\).
  \item \(\Delta \<1> = \<1> \otimes \mathbf{1} + \<Xb> \otimes \<X> + \mathbf{1} \otimes \<1>\).
  \item \(\Delta \<2> = \<2> \otimes \mathbf{1} + \<Xb> \<Xc> \otimes \<X> + \<Xc> \otimes \<1> 
    + \<Xb> \otimes \<1ac> + \mathbf{1} \otimes \<2>\).
\end{itemize}

\section{Hopf Algebra}

The algebra \(\langle \CF\rangle\) equipped with the coproduct \(\Delta\) forms what is known as a 
Hopf algebra (in particular, it is the Connes-Kreimer Hopf algebra). We in this section provide a 
informal introduction to Hopf algebras.

Before we proceed with defining Hopf algebra, we first need to introduce \textit{bialgebras}.
Suppose we have an algebra \(\CH^*\) acting on another algebra \(\CH^*\) via the action 
\[\langle \cdot, \cdot \rangle : \CH^* \otimes \CH \to \mathbb{R}.\]
Informally speaking, a bialgebra is then \(\CH\) equipped with a \textit{coproduct} \(\Delta : \CH \to \CH \otimes \CH\) 
which encodes this pairing and in particular, preserves the product structure of its \textit{coalgebra} \(\CH^*\). More precisely,
\(\Delta\) is dual to the product operator \(\nabla^*\) of \(\CH^*\) in the sense that 
\[\langle \nabla^*(f \otimes g), h\rangle = \langle f \otimes g, \Delta h\rangle\]
where \(\nabla^* : \CH^* \otimes \CH^* \to \CH^*\) is such that \(\nabla^*(x, y) = xy\). Thus, in some 
sense, once we have the coproduct \(\Delta\), we can forget about \(\CH^*\) and work solely with \(\CH\).

A bialgebra \((\CH, \Delta)\) is \textit{graded} if it has the decomposition 
\[\CH = \bigoplus_{n \in \mathbb{N}} \CH_{(n)}\]
where \(H_{(n)}\) are vector spaces such that for all \(n, m \in \mathbb{N}\),
\[\CH_{(n)} \cdot \CH_{(m)} \subseteq \CH_{(n + m)}, \text{ and } 
  \Delta \CH_{(n)} \subseteq \bigoplus_{p + q = n} \CH_{(p)} \otimes \CH_{(q)}.\]
Denoting \(\CF_{(n)} = \{f \in \CF : |f| = n\}\), taking \(\CH_{(n)} = \langle \CF_{(n)}\rangle\), 
we observe that \(\CF\) is a graded bialgebra. 

While the coproduct preserves the product structure of \(\CH^*\), we would also like a map which 
preserves the units (invertible elements) of \(\CH^*\). This is achieved with the \textit{antipole} 
map \(S : \CH \to \CH\) which is required to satisfy
\begin{equation}\label{eq:antipole}
  \nabla (S \otimes \text{id}) \Delta h = \nabla (\text{id} \otimes S) \Delta h = \langle \mathbf{1}, h\rangle \mathbf{1}
\end{equation}
for all \(h \in \CH\), where \(\nabla\) is the product operator of \(\CH\) and 
\[\langle\mathbf{1}, h\rangle = 
\begin{cases}
  1 & \text{if } h = \mathbf{1},\\
  0 & \text{otherwise}.
\end{cases}\]
This condition is motivated by the fact that, if \(S\) is such that for any unit \(f \in \CH^*\), 
\(S^* f = f^{-1}\) where \(S^* : \CH^* \to \CH^*\) is the dual of \(S\), then
\[\langle f \otimes f, (S \otimes \text{id})\Delta h\rangle 
  = \langle \nabla^* (S^* f \otimes f), h\rangle = \langle S^* f \cdot f, h\rangle = \langle \mathbf{1}, h\rangle.\]
On the other hand, 
\[\langle f \otimes f, \langle \mathbf{1}, h\rangle \mathbf{1} \otimes \mathbf{1}\rangle = \langle \mathbf{1}, h\rangle.\]
Hence, in order for \(S\) to preserve the unit structure of \(\CH^*\), it is necessary to require
\(\nabla (S \otimes \text{id})\Delta h = \nabla \langle \mathbf{1}, h\rangle \mathbf{1} \otimes \mathbf{1} 
= \langle \mathbf{1}, h\rangle\). 

With the antipole motivated, a Hopf algebra is then simply a bialgebra equipped with an antipole.

\begin{proposition}
  A graded bialgebra satisfying \(\CH_{(0)} = \mathbb{R}\) is automatically a Hopf algebra.
\end{proposition}

Thus, as \(\langle \CF_{(0)}\rangle = \mathbb{R}\), it follows \(\langle \CF\rangle\) is a Hopf algebra.
The antipole \(S\) of \(\langle \CF\rangle\) can be computed explicitly by using the identity~\eqref{eq:antipole}. 
We give some examples:
\begin{itemize}
  \item \(S \mathbf{1} = \mathbf{1}\).
  \item \(0 = \nabla (S \otimes \text{id}) \Delta \<X> = S \<X> + \<X>\). Thus, \(S \<X> = -\<X>\).
  \item \(0 = \nabla (S \otimes \text{id}) \Delta \<1> = S \<1> + (S \<Xb>) \<X> + \<1>\). 
    Thus, \(S \<1> = -\<1> + \<X> \<Xb>\).
  \item \(0 = \nabla (S \otimes \text{id}) \Delta \<2> = S \<2> + (S \<Xb> \<Xc>) \<X> + (S \<Xc>) \<1> + (S \<Xb>) \<1ac> + \<2>\).
    Thus, \(S \<2> = -\<2> + \<Xc> \<1> + \<Xb> \<1ac> - \<X> \<Xb> \<Xc>\).
\end{itemize}

Equipping \(\langle \CF \rangle\) with the inner product as defined by Equation~\eqref{eq:inner-product}, 
we can view \(\langle \CF \rangle\) acting on itself via the action \(\langle f, \cdot \rangle\) 
for all \(f \in \langle \CF \rangle\). Thus, the underlying bialgebra of \((\langle \CF \rangle, \Delta, S)\) 
can be viewed as \(\CH = \langle \CF \rangle\) with itself as its own coalgebra \(\CH^* = \langle \CF \rangle\) 
(with a different multiplication). We now give a description of the algebraic structure of 
\(\CH^* = \langle \CF \rangle\).

By the property of the coproduct, if \(\CH^* = \langle \CF \rangle\) has product \(\nabla^*(f, g) = f * g\), then
for any \(f, g \in \CF\) and \(h \in \CF\), we have the defining property for \(\nabla^*\):
\[\langle f * g, h\rangle = \langle \nabla^*(f, g), h\rangle = \langle f \otimes g, \Delta h\rangle.\]
The product \(*\) is known as the \textit{convolution product} and we observe that for \(f, g \in \CF\), 
\(\langle f * g, h\rangle = 1\) if and only if \(f \otimes g\) is a term of \(\Delta h\). For trees, the 
convolution can be explicitly described by
\[\tau_1 * \tau_2 = \tau_1 \tau_2 + \tau_1 *_t \tau_2\]
for \(\tau_1, \tau_2 \in \CT\) where \(\tau_1 *_t \tau_2\) denotes the sum of all trees obtained 
by attaching \(\tau_1\) to a vertex of \(\tau_2\).

\(\langle \CF \rangle\) equipped with \(*\) forms an associative (\textit{non-commutative}) algebra 
and we denote its coproduct by \(\delta :  \langle \CF \rangle \to  \langle \CF \rangle \otimes  \langle \CF \rangle\) 
such that 
\[\langle \delta f, h_1 \otimes h_2\rangle = \langle f, h_1 h_2\rangle.\]
Thus, if \(\tau \in \CT\), then, using Sweedler's notation,
\[1 = \langle \tau, \tau \cdot \mathbf{1}\rangle = \langle \delta \tau, \tau \otimes \mathbf{1}\rangle
  = \sum \langle \tau^{(1)}, \tau\rangle \langle \tau^{(2)}, \mathbf{1}\rangle.\]
Namely, there exists a unique component \(i\) of \(\delta \tau\) for which \(\tau^{(1)i} = \tau\) and 
\(\tau^{(2)i} = \mathbf{1}\).

The tensor algebra \(T(\mathbb{R}^d)\) can be identified in \(\langle \CF \rangle\) by identifying 
\(e_a\) by \(\<X>\), \(e_b \otimes e_a\) by \(\<1>\), \(e_c \otimes e_b \otimes e_a\) by \(\<10>\)
and so on for \(a, b, c = 1, \dots, d\). Namely, denoting 
\(\iota : T(\mathbb{R}^d) \hookrightarrow \langle \CF \rangle\) for this inclusion, 
\(\iota(T(\mathbb{R}^d))\) corresponds to the set of all trees with a single branch.

\begin{definition}
  Let \(\CH\) be a Hopf algebra with coalgebra \(\CH^*\). Denote \(\hom(\CH, \mathbb{R}) \subseteq \CH^*\) 
  for the the space of characters (i.e. \(\mathbb{R}\)-valued algebra homomorphisms) of \(\CH\).
\end{definition}

For all (by an abuse of notation) \(f = \langle f, \cdot \rangle \in \hom(\CH, \mathbb{R})\), 
as \(f\) is multiplicative, it follows that 
\[\langle f, h_1 h_2\rangle = \langle f, h_1\rangle \langle f, h_2\rangle = 
  \langle f \otimes f, h_1 \otimes h_2\rangle.\]
On the other hand, denoting \(\delta\) for the coproduct on the coalgebra \(\CH^*\) (the cocoproduct?), we have 
\[\langle \delta f, h_1 \otimes h_2\rangle = \langle f, h_1 h_2\rangle.\] 
Thus, we can identify elements of \(\hom(\CH, \mathbb{R})\) with \(f \in \CH^*\) which satisfies 
\(\delta f = f \otimes f\). More precisely, 
\[\hom(\CH, \mathbb{R}) \simeq \{f \in \CH^* : \delta f = f \otimes f\} =: G(\CH).\]

\begin{definition}
  \(G(\CH)\) forms a group and is known as the \textit{Butcher group}.
\end{definition}

\begin{proposition}
  For all \(f \in G(\CH)\), \(f^{-1} = S^* f\) with \(S^*\) being the adjoint of the antipole.
\end{proposition}

\section{Branched Rough Paths}

Finally, we can now define branched rough paths. Defining \(\CH^*_{(n)} = \langle \CF_{(n)}\rangle\) 
where the notation \(^*\) indicates we are working in the dual space, we define 
\[G_N(\CH) = \frac{G(\CH)}{G(\CH) \cap \bigoplus_{n \ge N + 1} \CH^*_{(n)}}\]
where the fraction indicates the quotient group (it is not difficult the see that the denominator 
is indeed a normal subgroup). Namely, we identify the forests in \(G(\CH)\) which has order greater 
than \(N\) with the identity element.

\begin{definition}
  For a map \(\mathbf{X} : [0, T] \to G_N(\CH)\), we denote \(\mathbf{X}_{st} = \mathbf{X}^{-1}_s * \mathbf{X}_t\).
  \(\mathbf{X}\) is a \(\gamma\)-H\"older branched rough path if 
  \[\sup_{s < t \in [0, T]} \frac{|\langle \mathbf{X}_{st}, \tau\rangle|} {|t - s|^{\gamma|\tau|}} < \infty\]
  for all \(\tau \in \CF_N\).

  For a branched rough path \(\mathbf{X}\), we define its path component \(X^a\) by 
  \[\delta X^a_{st} = \langle \mathbf{X}_{st}, \<X>\rangle.\]
\end{definition}

We observe that the path components of a \(\gamma\)-H\"older branched rough path is \(\gamma\)-H\"older. 
Moreover, Chen's condition (that is \(\mathbf{X}_{st} = \mathbf{X}_{su} * \mathbf{X}_{ut}\)) is automatically 
satisfied by branching rough paths. 

For general trees \(\tau = [f]_a\), we interpret \(\langle \mathbf{X}_{st}, \tau\rangle = 
  \int_s^t \langle \mathbf{X}_{sr}, f\rangle \dd X^a_r\) (note that here, the right hand side is strictly 
formal and is simply a notation for the left hand side). Thus, 
\[\langle \mathbf{X}_{st}, \<X> \rangle = \int_s^t \dd X^a_r = \delta X^a_{st}, \qquad
  \langle \mathbf{X}_{st}, \<1> \rangle = \int_s^t \int_s^r \dd X^b_u \dd X^a_r,\]
\[\langle \mathbf{X}_{st}, \<2>\rangle = \int_s^t \left(\int_s^r \dd X^b_u\right) \left(\int_s^r \dd X^c_u\right) \dd X^a_r,\] 
and so on.

\begin{definition}
  A path \(\mathbf{Y} : [0, T] \to \CH_{N - 1} = \bigoplus_{n = 0}^{N - 1} \CH_{(n)}\) is a 
  \(\mathbf{X}\)-controlled rough path for some \(\gamma\)-H\"older branched rough path \(\mathbf{X}\) if 
  \[R^h_{st} = \langle h, \mathbf{Y}_t\rangle - \langle \mathbf{X}_{st} * h, \mathbf{Y}_s\rangle\]
  satisfies 
  \[\sup_{s < t \in [0, T]} \frac{|R^h_{st}|}{|t - s|^{(N - |h|)\gamma}} < \infty\] 
  for all \(h \in \CF_{N - 1}\). We denote \(Y_t = \langle \mathbf{1}, \mathbf{Y}_t\rangle\) for the 
  path component of \(\mathbf{Y}\). Moreover, for \(h \in \CF_{N - 1}\), denote 
  \(\partial^h_X Y_t = \langle h, \mathbf{Y}_t\rangle\). We call \(\partial^h_X Y_t\) the \(h\)-th Gubinelli 
  derivative of \(\mathbf{Y}\) against \(\mathbf{X}\).
\end{definition}

We now discuss the case where there is only one label and we omit it from the notations.

This definition corresponds to the usual definition of level 2 controlled rough paths in the following sense: 
assuming we are in 1-dimension and \(N = 2\), testing against \(h = \mathbf{1}\), we obtain
\[R^{\mathbf{1}}_{st} = Y_t - \langle \mathbf{X}_{st} * \mathbf{1}, \mathbf{Y}_s\rangle 
  = Y_t - \langle \mathbf{X}_{st}, \mathbf{Y}_s\rangle.\]
  % = Y_t - \langle \mathbf{X}_{st} \otimes \mathbf{1}, \Delta \mathbf{Y}_s\rangle.\]
% Then, writing \(\Delta \mathbf{Y}_s = \sum \mathbf{Y}^{(1)}_s \otimes \mathbf{Y}^{(2)}_s\), we have 
% \[\langle \mathbf{X}_{st} \otimes \mathbf{1}, \Delta \mathbf{Y}_s\rangle
%   = \sum \langle \mathbf{X}_{st}, \mathbf{Y}^{(1)}_s\rangle \mathbf{1}_{\mathbf{1} = \mathbf{Y}^{(2)}_s} 
%   = \langle \mathbf{X}_{st}, \mathbf{Y}_s\rangle\]
% since the \(\mathbf{Y}^{(2)}_s = \mathbf{1}\) if and only if \(\mathbf{Y}^{(1)}_s = \mathbf{Y}_s\). 
Now, as 
\begin{align*}
  \langle \mathbf{X}_{st}, \mathbf{Y}_s\rangle 
  & = \sum_{f \in \CF_{1}} \langle f, \mathbf{Y}_s\rangle \langle \mathbf{X}_{st}, f\rangle
    = Y_s \langle \mathbf{X}_{st}, \mathbf{1}\rangle 
      + \langle \<X'>, \mathbf{Y}_s\rangle \langle \mathbf{X}_{st}, \<X'>\rangle\\
  & = Y_s + \langle \<X'>, \mathbf{Y}_s\rangle \delta X_{st}
\end{align*}
where the last equality follows as \(\mathbf{X}_{st} \in \hom(\CH, \mathbb{R})\) so 
\(\langle \mathbf{X}_{st}, \mathbf{1}\rangle = 1\). Thus, the Gubinelli derivative of \(Y\) against 
\(X\) in this case is \(Y'_s = \partial^{\<X'>}_X Y_s = \langle \<X'>, \mathbf{Y}_s\rangle\). 
More generally, for any \(N\), we find that
\begin{align*}
  R^{\mathbf{1}}_{st} 
  & = \delta Y_{st} - 
  \sum_{f \in \CF_{N - 1} \setminus \{\mathbf{1}\}} \partial_X^f Y_s \langle \mathbf{X}_{st}, f\rangle\\
  & = \delta Y_{st} - Y'_s \delta X_{st} - 
    \sum_{f \in \CF_{N - 1} \setminus \{\mathbf{1}, \<X'>\}} \langle f, \mathbf{Y}_s\rangle \langle \mathbf{X}_{st}, f\rangle
\end{align*}
\begin{proposition}
  Let \(\mathbf{Y} : [0, T] \to \CH^*_{N - 1}\) be a controlled rough path against the \(\gamma\)-H\"older 
  rough path \(\mathbf{X}\). Then, denoting the two parameter process
  \[I_{st}^a = \sum_{f \in \CF_{N - 1}} \langle f, \mathbf{Y}_s\rangle \langle \mathbf{X}_{st}, [f]_a\rangle\] 
  we have that 
  \[\|\delta I\|_{(N + 1)\gamma} \lesssim \sum_{f \in \CF_{N - 1}} \|R^f\|_{(N - |f|)\gamma} \|\mathbf{X}^{[f]}\|_{(|f| + 1)\gamma}\]
  where we denote \(\mathbf{X}^f = \langle \mathbf{X}, f\rangle\).
\end{proposition}
\begin{proof}
  Firstly, we observe that, for all \(f \in \CF_{N - 1}\),
  \begin{align*}
    \langle\mathbf{X}_{st}, [f]\rangle - \langle \mathbf{X}_{su}, [f]\rangle 
    % & = \langle \mathbf{X}_{su} * \mathbf{X}_{ut}, [f]\rangle - \langle \mathbf{X}_{su}, [f]\rangle\\
    % & = \langle \mathbf{X}_{su} \otimes \mathbf{X}_{ut}, \Delta [f]\rangle - \langle \mathbf{X}_{su}, [f]\rangle\\
    & = \langle \mathbf{X}_{su} \otimes \mathbf{X}_{ut}, (\text{id} \otimes [-]) \Delta f\rangle\\
    & = \langle \nabla (\mathbf{X}_{su} \otimes [-]^* \mathbf{X}_{ut}), f\rangle\\ 
    & = \langle \mathbf{X}_{su}, f\rangle \langle \mathbf{X}_{ut}, [f]\rangle.
  \end{align*}
  Thus, adopting Einstein's summation convention, we have
  \begin{align*}
    \delta I_{sut} & = \langle f, \mathbf{Y}_s\rangle(\langle\mathbf{X}_{st}, [f]\rangle - \langle \mathbf{X}_{su}, [f]\rangle) 
      - \langle f, \mathbf{Y}_u\rangle \langle \mathbf{X}_{ut}, [f]\rangle\\
    & = (\langle f, \mathbf{Y}_s\rangle \langle \mathbf{X}_{su}, f\rangle - \langle f, \mathbf{Y}_u\rangle) 
        \langle \mathbf{X}_{ut}, [f]\rangle\\
    & = - R^f_{su} \langle \mathbf{X}_{ut}, [f]\rangle 
      + \underbrace{(\langle f, \mathbf{Y}_s\rangle \langle \mathbf{X}_{su}, f\rangle -
          \langle \mathbf{X}_{su} * f, \mathbf{Y}_s\rangle)}_{\text{?}}
        \langle \mathbf{X}_{ut}, [f]\rangle\\
    & = \cdots
  \end{align*}
  % Hence, we may conclude as each summand has order \(|u - s|^{(N - |f|)\gamma} |t - u|^{|f|\gamma} \le |t - s|^{N\gamma}\), 
  % and the sum is finite.
\end{proof}

As a consequence of the above proposition, in the case where \((N + 1) \gamma > 1\), we may apply the sewing 
lemma to \(I_{st}\) to obtain 
\[\int_s^t Y_r \dd X_r := \lim_{|\CP| \to 0} \sum_{(u, v) \in \CP} I_{uv}\]
with \(\CP\) being a partition of \([s, t]\) and we have the remainder inequality 
\begin{align*}
  & \left|\int_s^t Y_r \dd X_r - \sum_{f \in \CF_{N - 1}} \langle f, \mathbf{Y}_s\rangle \langle \mathbf{X}_{st}, [f]\rangle\right|\\
  \lesssim & \left(\sum_{f \in \CF_{N - 1}} \|R^f\|_{(N - |f|)\gamma} \|\mathbf{X}^{[f]}\|_{(|f| + 1)\gamma}\right)|t - s|^{(N + 1)\gamma}
\end{align*}
We note that at this point \(\int_s^{\cdot} Y_r \dd X_r\) takes value in \(\mathbb{R}\) and is notably not a tree/forest 
path. As we would like to talk about rough differential equations, namely we would want to phrase a fixed point 
argument involving rough integrals, we need to extend the \(\mathbb{R}\)-valued path \(\int_s^{\cdot} Y_r \dd X_r\) 
to a forest path \(\int_s^{\cdot} \mathbf{Y}_r \dd X_r\). This is achieved by specifying 
\(\partial^f_X \left(\int_s^{\cdot} \mathbf{Y}_r \dd X_r\right) = \left\langle f, \int_s^{\cdot} \mathbf{Y}_r \dd X_r \right\rangle\) 
for each \(f \in \CF_{N - 1}^*\) with 
\[\left\langle \mathbf{1}, \int_s^{\cdot} \mathbf{Y}_r \dd X_r \right\rangle = \int_s^{\cdot} Y_r \dd X_r.\]
For \(f \in \CF_{\le N - 2}\), we set 
\[\left\langle [f], \int_0^{\cdot} \mathbf{Y}_r \dd X_r \right\rangle = \langle f, \mathbf{Y}_{\cdot}\rangle.\]

\section{Composition with smooth functions}

Recall that, in the case where \(N = 2\), for \(\mathbf{X} = (X, \mathbb{X}) \in \CC^{\alpha}([0, T], V)\), 
\((Y, Y') \in \CD_X^{2\alpha}([0, T], W)\) and \(\phi \in C^2_b(W, \bar W)\), we have 
\[(\phi(Y), \phi(Y)') = (\phi(Y), D\phi(Y) Y') \in \CD_X^{2\alpha}([0, T], \bar W).\]
% \graybox{\begin{proof}
%   This follows directly by estimates on the remainder of Taylor expansion:
%   \begin{equs}
%     \|R^{\phi(Y)}_{s, t}\| & = \|\phi(Y_t) - \phi(Y_s) - D\phi(Y_s) Y'_s X_{s, t}\|\\
%     & = \|\phi(Y_t) - \phi(Y_s) - D\phi(Y_s) (Y_t - Y_s - R^Y_{s, t})\|\\
%     & \le \|D\phi(Y_s)\| \|R^Y_{s, t}\| + 
%         \frac{\|Y_{s, t}\|^2}{2} \sup_{\lambda \in [0, 1]} \|D^2\phi(\lambda Y_t + (1 - \lambda) Y_s)\|\\
%     & = O(|t - s|^{2\alpha}).
%   \end{equs}
% \end{proof}}\\
We in this section generalize this fact for branched rough paths.

\begin{proposition}
  Let \(\mathbf{X}\) be a level-\(N\) \(\gamma\)-H\"older rough path and \(\mathbf{Y}\) 
  be a \(d\)-dimensional controlled rough path against \(\mathbf{X}\). 
  Then, for any \(\phi \in C^{N - 1}_b(\mathbb{R}^d, \mathbb{R}^d)\), defining \(\phi(\mathbf{Y})\) by
  \[\phi(\mathbf{Y}) = \sum_{f \in \CF_{N - 1}} \partial^f_X \phi(\mathbf{Y}) f\]
  where 
  \[\partial^f_X \phi(\mathbf{Y}) = \sum_{\sigma \in \mathrm{Sym}(\# f)} \frac{1}{(\# f)!}D^{\# f} 
      \phi(Y)[(\partial^{h_{\sigma(i)}}_X Y)_{i = 1}^{\# f}]\]
  for any \(f = \prod_{i = 1}^{\# f} h_i \in \CF_{N - 1}\), we have that \(\phi(\mathbf{Y})\) is 
  also a controlled rough path against \(\mathbf{X}\).
\end{proposition}


\end{document}
